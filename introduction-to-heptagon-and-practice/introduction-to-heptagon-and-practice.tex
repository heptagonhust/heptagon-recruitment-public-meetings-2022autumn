\documentclass{beamer}
\usepackage[utf8]{inputenc}

\usepackage{xeCJK}

\usetheme{Madrid}
\usecolortheme{default}

\title{2022年秋季招新线上培训}
\subtitle{超算队介绍及题目背景讲解}
\author{七边形超算队}
\institute{华中科技大学}
\date{2022年11月19日}
\logo{\includegraphics[height=1cm]{heptagon.jpg}}
\begin{document}

\AtBeginSection[]
{
  \begin{frame}
    \frametitle{目录}
    \tableofcontents[currentsection]
  \end{frame}
}

\begin{document}

%The next statement creates the title page.
\frame{\titlepage}

%---------------------------------------------------------
%This block of code is for the table of contents after
%the title page
\begin{frame}
\frametitle{目录}
\tableofcontents
\end{frame}
%---------------------------------------------------------


\section{关于七边形超算队}

%---------------------------------------------------------
%Changing visivility of the text
\begin{frame}
\frametitle{简介}

\begin{itemize}
    \item<1-> 七边形超算队是一支专注于计算机系统与高性能计算的队伍,主要由本科生组成,并由计算机科学与技术学院石宣化教授担任指导老师  
    \item<2-> 七边形超算队建立于2013年,至今已经获得过国内国际许多高性能计算赛事的奖项,有很多优秀的学长学姐 
    \item<3-> 欢迎对计算机系统感兴趣,想要参加比赛的同学加入七边形
    
\end{itemize}

\end{frame}

\begin{frame}
\frametitle{日常}
\begin{itemize}
    \item<1-> 参加比赛
    \item<2-> 代码训练
    \item<3-> 集群运维
\end{itemize}
\end{frame}

\begin{frame}
\frametitle{成长方向}
\begin{itemize}
    \item<1-> 计算机系统/高性能计算
    \item<2-> 分布式计算
    \item<3-> 机器学习
    \item<4-> 异构处理器
\end{itemize}
\end{frame}

\section{关于超算竞赛}

%---------------------------------------------------------
%Highlighting text
\begin{frame}
\frametitle{比赛内容}

\begin{itemize}
    \item<1-> 集群软硬件环境搭建与调试
    \item<2-> 优化/重构baseline代码,保证精度,提高性能
\end{itemize}

\end{frame}

\begin{frame}
\frametitle{赛事特点}

\begin{itemize}
    \item<1-> 赛事少
    \item<2-> 周期长
    \item<3-> 工作量大
    \item<4-> 竞争对手背景多样
\end{itemize}

\end{frame}

\begin{frame}
\frametitle{涉及知识}

\begin{itemize}
    \item<1-> 计算机系统
    \item<2-> 并行/高性能计算
    \item<3-> 异构编程
\end{itemize}

\end{frame}


\section{招新练习题目}

\begin{frame}
\frametitle{背景介绍}
\href{https://github.com/heptagonhust/bicubic-image-resize}{\underline{基于双三次插值法的图像缩放}} \pause

\begin{itemize}
    \item<1-> 边界处理
    \item<2-> 不允许修改的部分
    \item<3-> 运行与评测
\end{itemize}
\end{frame}


\section{答疑}

\begin{frame}{Frame Title}
    
\item 招新标准是什么?

\item

\item 基础不好怎么办?

\item

\item ... 

\end{frame}

\end{document}